% Options for packages loaded elsewhere
\PassOptionsToPackage{unicode}{hyperref}
\PassOptionsToPackage{hyphens}{url}
%
\documentclass[
]{article}
\usepackage{amsmath,amssymb}
\usepackage{iftex}
\ifPDFTeX
  \usepackage[T1]{fontenc}
  \usepackage[utf8]{inputenc}
  \usepackage{textcomp} % provide euro and other symbols
\else % if luatex or xetex
  \usepackage{unicode-math} % this also loads fontspec
  \defaultfontfeatures{Scale=MatchLowercase}
  \defaultfontfeatures[\rmfamily]{Ligatures=TeX,Scale=1}
\fi
\usepackage{lmodern}
\ifPDFTeX\else
  % xetex/luatex font selection
\fi
% Use upquote if available, for straight quotes in verbatim environments
\IfFileExists{upquote.sty}{\usepackage{upquote}}{}
\IfFileExists{microtype.sty}{% use microtype if available
  \usepackage[]{microtype}
  \UseMicrotypeSet[protrusion]{basicmath} % disable protrusion for tt fonts
}{}
\makeatletter
\@ifundefined{KOMAClassName}{% if non-KOMA class
  \IfFileExists{parskip.sty}{%
    \usepackage{parskip}
  }{% else
    \setlength{\parindent}{0pt}
    \setlength{\parskip}{6pt plus 2pt minus 1pt}}
}{% if KOMA class
  \KOMAoptions{parskip=half}}
\makeatother
\usepackage{xcolor}
\usepackage[margin=1in]{geometry}
\usepackage{color}
\usepackage{fancyvrb}
\newcommand{\VerbBar}{|}
\newcommand{\VERB}{\Verb[commandchars=\\\{\}]}
\DefineVerbatimEnvironment{Highlighting}{Verbatim}{commandchars=\\\{\}}
% Add ',fontsize=\small' for more characters per line
\usepackage{framed}
\definecolor{shadecolor}{RGB}{248,248,248}
\newenvironment{Shaded}{\begin{snugshade}}{\end{snugshade}}
\newcommand{\AlertTok}[1]{\textcolor[rgb]{0.94,0.16,0.16}{#1}}
\newcommand{\AnnotationTok}[1]{\textcolor[rgb]{0.56,0.35,0.01}{\textbf{\textit{#1}}}}
\newcommand{\AttributeTok}[1]{\textcolor[rgb]{0.13,0.29,0.53}{#1}}
\newcommand{\BaseNTok}[1]{\textcolor[rgb]{0.00,0.00,0.81}{#1}}
\newcommand{\BuiltInTok}[1]{#1}
\newcommand{\CharTok}[1]{\textcolor[rgb]{0.31,0.60,0.02}{#1}}
\newcommand{\CommentTok}[1]{\textcolor[rgb]{0.56,0.35,0.01}{\textit{#1}}}
\newcommand{\CommentVarTok}[1]{\textcolor[rgb]{0.56,0.35,0.01}{\textbf{\textit{#1}}}}
\newcommand{\ConstantTok}[1]{\textcolor[rgb]{0.56,0.35,0.01}{#1}}
\newcommand{\ControlFlowTok}[1]{\textcolor[rgb]{0.13,0.29,0.53}{\textbf{#1}}}
\newcommand{\DataTypeTok}[1]{\textcolor[rgb]{0.13,0.29,0.53}{#1}}
\newcommand{\DecValTok}[1]{\textcolor[rgb]{0.00,0.00,0.81}{#1}}
\newcommand{\DocumentationTok}[1]{\textcolor[rgb]{0.56,0.35,0.01}{\textbf{\textit{#1}}}}
\newcommand{\ErrorTok}[1]{\textcolor[rgb]{0.64,0.00,0.00}{\textbf{#1}}}
\newcommand{\ExtensionTok}[1]{#1}
\newcommand{\FloatTok}[1]{\textcolor[rgb]{0.00,0.00,0.81}{#1}}
\newcommand{\FunctionTok}[1]{\textcolor[rgb]{0.13,0.29,0.53}{\textbf{#1}}}
\newcommand{\ImportTok}[1]{#1}
\newcommand{\InformationTok}[1]{\textcolor[rgb]{0.56,0.35,0.01}{\textbf{\textit{#1}}}}
\newcommand{\KeywordTok}[1]{\textcolor[rgb]{0.13,0.29,0.53}{\textbf{#1}}}
\newcommand{\NormalTok}[1]{#1}
\newcommand{\OperatorTok}[1]{\textcolor[rgb]{0.81,0.36,0.00}{\textbf{#1}}}
\newcommand{\OtherTok}[1]{\textcolor[rgb]{0.56,0.35,0.01}{#1}}
\newcommand{\PreprocessorTok}[1]{\textcolor[rgb]{0.56,0.35,0.01}{\textit{#1}}}
\newcommand{\RegionMarkerTok}[1]{#1}
\newcommand{\SpecialCharTok}[1]{\textcolor[rgb]{0.81,0.36,0.00}{\textbf{#1}}}
\newcommand{\SpecialStringTok}[1]{\textcolor[rgb]{0.31,0.60,0.02}{#1}}
\newcommand{\StringTok}[1]{\textcolor[rgb]{0.31,0.60,0.02}{#1}}
\newcommand{\VariableTok}[1]{\textcolor[rgb]{0.00,0.00,0.00}{#1}}
\newcommand{\VerbatimStringTok}[1]{\textcolor[rgb]{0.31,0.60,0.02}{#1}}
\newcommand{\WarningTok}[1]{\textcolor[rgb]{0.56,0.35,0.01}{\textbf{\textit{#1}}}}
\usepackage{graphicx}
\makeatletter
\newsavebox\pandoc@box
\newcommand*\pandocbounded[1]{% scales image to fit in text height/width
  \sbox\pandoc@box{#1}%
  \Gscale@div\@tempa{\textheight}{\dimexpr\ht\pandoc@box+\dp\pandoc@box\relax}%
  \Gscale@div\@tempb{\linewidth}{\wd\pandoc@box}%
  \ifdim\@tempb\p@<\@tempa\p@\let\@tempa\@tempb\fi% select the smaller of both
  \ifdim\@tempa\p@<\p@\scalebox{\@tempa}{\usebox\pandoc@box}%
  \else\usebox{\pandoc@box}%
  \fi%
}
% Set default figure placement to htbp
\def\fps@figure{htbp}
\makeatother
\setlength{\emergencystretch}{3em} % prevent overfull lines
\providecommand{\tightlist}{%
  \setlength{\itemsep}{0pt}\setlength{\parskip}{0pt}}
\setcounter{secnumdepth}{-\maxdimen} % remove section numbering
% ======================
% PACKAGE CONFIGURATION
% ======================
\usepackage[utf8]{inputenc}

% Math
\usepackage{amsmath, amssymb, mathtools}
\usepackage{cancel}

% Graphics and TikZ
\usepackage{graphicx}
\usepackage{tikz}
\usetikzlibrary{calc, positioning, decorations.pathreplacing, arrows.meta}

% Layout
\usepackage{geometry}
\geometry{margin=0.75in}
\usepackage{fancyhdr}
\setlength{\headheight}{14.4pt}        % Fix for fancyhdr header size warning
\addtolength{\topmargin}{-2.4pt}       % Optional: counteract added header height

% Fonts and styling
\usepackage[fontsize=14pt]{scrextend}
\usepackage{anyfontsize}
\usepackage{bm}
\usepackage{xfakebold}
\usepackage{xspace}

% Lists and layout
\usepackage{array}
\usepackage{enumitem}
\usepackage{wrapfig}
\usepackage{makecell}
\usepackage{booktabs}

% Boxes and framing
\usepackage{mdframed}
\usepackage{tcolorbox}

% Document structure
\usepackage{titlesec}
\usepackage{extramarks} % [?] Only needed if using \extramarks
\usepackage{caption}    % [?] Only needed if customizing captions

% Colors
\usepackage[dvipsnames]{xcolor}

% Utility
\usepackage{pifont}
\usepackage{silence}
\usepackage{xparse}

% Hyperlinks (should be last)
\usepackage{hyperref}


% ======================
% COLOR DEFINITIONS
% ======================
\definecolor{mygreen}{RGB}{131, 176, 47}
\definecolor{textgreen}{RGB}{129, 186, 24}
\definecolor{myred}{RGB}{195, 30, 74}
\definecolor{myblue}{RGB}{0, 140, 158}
\definecolor{myorange}{RGB}{240, 110, 41}
\definecolor{mygrey}{RGB}{156, 156, 156}
\definecolor{dark}{HTML}{83B02F}
\definecolor{light}{HTML}{e1f0c7}

% ======================
% SPACING ADJUSTMENTS
% ======================
\setlength{\abovedisplayskip}{5pt}
\setlength{\belowdisplayskip}{5pt}
\setlength{\abovedisplayshortskip}{1pt}
\setlength{\belowdisplayshortskip}{1pt}

% ======================
% CUSTOM ENVIRONMENTS
% ======================
\newenvironment{answer}{
  \begin{list}{}{
    \setlength{\leftmargin}{2cm}
    \setlength{\rightmargin}{0cm}
    \setlength{\listparindent}{0cm}
    \setlength{\itemindent}{0cm}
    \setlength{\parsep}{0pt}
  }
  \item[]
}{
  \end{list}
}

% ======================
% CUSTOM COMMANDS
% ======================
\newcommand{\tikzmark}[1]{\tikz[remember picture, baseline] \node (#1) {};}
\newcommand{\mathbox}[2]{%
  \begingroup
  \setlength{\fboxsep}{4pt}
  \setlength{\fboxrule}{1.2pt}
  \color{#1}
  \fbox{\color{black} $\displaystyle #2$}
  \endgroup
}
\newcommand{\dash}{-\mkern-8mu-}
\newcommand{\mytitle}{TITLE PLACEHOLDER}
\newcommand*{\underuparrow}[1]{\ensuremath{\underset{\uparrow}{#1}}}
\newcommand{\fbseries}{\unskip\setBold\aftergroup\unsetBold\aftergroup\ignorespaces}
\makeatletter
\newcommand{\setBoldness}[1]{\def\fake@bold{#1}}
\makeatother
\newcommand{\highlight}[1]{%
    \textcolor{textgreen}{\fbseries #1}\xspace%
}
\newcommand{\elbowarrow}{%
  \tikz[baseline=-0.5ex, x=0.4ex, y=0.4ex, line width=0.4pt, line join=round]{
    \def\shaftlen{3}
    \draw
      (-0.25,0) -- (-0.25,\shaftlen) -- (1,\shaftlen)
      -- (1,1) -- (\shaftlen,1)
      -- (\shaftlen,1.75) -- ({\shaftlen + 1.5},0.5) -- (\shaftlen,-0.75) -- (\shaftlen,0)
      -- (1,0) -- cycle;
  }%
}
\newcommand{\tallbar}{\makebox[0pt][r]{\raisebox{-2em}[0pt][0pt]{\rule{0.5pt}{3.3em}}}}

\newcolumntype{N}{|>{\raggedleft\arraybackslash}p{1.5em}} % Step number
\newcolumntype{S}{|>{\raggedright\arraybackslash}p{0.45\textwidth}} % Statement
\newcolumntype{J}{|>{\raggedright\arraybackslash}p{0.45\textwidth}|} % Justification
\newcolumntype{C}{|>{\raggedright\arraybackslash}p{0.45\textwidth}|}  % Conclusion (no vline)
\newcommand{\btherefore}{\raisebox{-0.65ex}{\(\bullet\)}\kern-0.13ex \raisebox{0.65ex}{\(\bullet\)}\kern-0.13ex \raisebox{-0.65ex}{\(\bullet\)}\kern-0.1em}

% ======================
% LIST AND SECTION FORMATTING
% ======================
\renewcommand{\labelitemii}{$\triangleright$}
\titleformat{\subsection}[runin]{\normalfont\large\bfseries}{\thesubsection}{1em}{}
\renewcommand\CancelColor{\color{red}}

% ======================
% HEADER CONFIGURATION
% ======================
\newcommand{\headertext}{}
\newlength{\headertextwidth}
\newcommand{\autoscaleheader}[1]{%
    \settowidth{\headertextwidth}{#1}%
    \ifdim\headertextwidth>0.55\textwidth
        \renewcommand{\headertext}{\scalebox{0.85}{#1}}%
        \fancyhead[L]{}%
    \else
        \renewcommand{\headertext}{#1}%
        \fancyhead[L]{\today}%
    \fi
}

% ======================
% FIRST PAGE STYLE
% ======================
\fancypagestyle{firstpage}{
  \fancyhf{}
  \fancyhead[L]{Aidan J. Wagner}
  \fancyhead[R]{\today}
  \fancyhead[C]{\mytitle}
  \renewcommand{\headrule}{%
      \hrule height 0.4pt \vspace{3pt}
      \rule{0.5\textwidth}{0.4pt} \\[-13pt]
      \rule{0.25\textwidth}{0.4pt}
      \vspace{-10pt}
  }
}

% ======================
% HEADER/FOOTER SETUP
% ======================
\pagestyle{fancy}
\fancyhf{}
\fancyhead[R]{\thepage}
\fancyhead[C]{\normalsize \headertext}
\fancyhead[L]{\today}
\renewcommand{\headrule}{%
    \hrule height 0.4pt \vspace{3pt}
    \rule{0.5\textwidth}{0.4pt} \\[-13pt]
    \rule{0.25\textwidth}{0.4pt}
    \vspace{-10pt}
}
\fancyfoot{}

% ======================
% CUSTOM HEADING LOGIC
% ======================
\makeatletter
\newcommand{\getcleansection}{%
    \expandafter\getcleansection@aux\leftmark\@nil
}
\def\getcleansection@aux#1\@nil{#1}

\newcommand{\@UNsection}[2]{%
    \section{\scalebox{#2}[#2]{#1}}%
    \markboth{\normalsize #1}{}%
    \autoscaleheader{{\normalsize #1 \quad - \quad}}%
    \ifnum\value{page}=1\else
        \fancyhead[C]{\headertext}%
        \thispagestyle{fancy}%
    \fi
}

\newcommand{\UNsection}{%
    \@ifnextchar\bgroup
        {\@UNsectionWithScale}
        {\@UNsectionDefault}
}
\newcommand{\@UNsectionWithScale}[1]{%
    \@ifnextchar\bgroup
        {\@UNsectionWithScaleAndTitle{#1}}
        {\@UNsection{#1}{1.0}}%
}
\newcommand{\@UNsectionWithScaleAndTitle}[2]{%
    \@UNsection{#1}{#2}%
}
\newcommand{\@UNsectionDefault}[1]{%
    \@UNsection{#1}{1.0}%
}

\newcommand{\@UNsubsection}[2]{%
    \subsection{\scalebox{#2}[#2]{#1}}%
    \markright{#1}%
    \autoscaleheader{{\getcleansection \quad - \quad #1}}%
    \ifnum\value{page}=1\else
        \fancyhead[C]{\headertext}%
        \thispagestyle{fancy}%
    \fi
}
\newcommand{\UNsubsection}{%
    \@ifnextchar\bgroup
        {\@UNsubsectionWithScale}
        {\@UNsubsectionDefault}
}
\newcommand{\@UNsubsectionWithScale}[1]{%
    \@ifnextchar\bgroup
        {\@UNsubsectionWithScaleAndTitle{#1}}
        {\@UNsubsection{#1}{1.0}}%
}
\newcommand{\@UNsubsectionWithScaleAndTitle}[2]{%
    \@UNsubsection{#1}{#2}%
}
\newcommand{\@UNsubsectionDefault}[1]{%
    \@UNsubsection{#1}{1.0}%
}

% ======================
% MULTIHEADING
% ======================
\newcommand{\@multiheading}[3]{%
    \section*{\Large\scalebox{#3}[#3]{#1} \hspace{0.25em} - \textcolor{mygrey}{\large #2}}%
    \vspace{-1em}%
    \markboth{#1}{#2}%
    \autoscaleheader{{#1 \quad - \quad #2}}%
    \ifnum\value{page}=1\else
        \fancyhead[C]{\headertext}%
        \thispagestyle{fancy}%
    \fi
}
\newcommand{\multiheading}{%
    \@ifnextchar\bgroup
        {\@multiheadingWithArgs}
        {\@multiheadingError}
}
\newcommand{\@multiheadingWithArgs}[1]{%
    \@ifnextchar\bgroup
        {\@multiheadingWithSubtitle{#1}}
        {\@multiheading{#1}{}{1.0}}%
}
\newcommand{\@multiheadingWithSubtitle}[2]{%
    \@ifnextchar\bgroup
        {\@multiheadingWithScale{#1}{#2}}
        {\@multiheading{#1}{#2}{1.0}}%
}
\newcommand{\@multiheadingWithScale}[3]{%
    \@multiheading{#1}{#2}{#3}%
}
\newcommand{\@multiheadingError}{%
    \PackageError{multiheading}{Invalid use of \string\multiheading}{Expected two or three arguments.}%
}
\makeatother
\renewcommand{\mytitle}{SnazzieR Example Tables}
\usepackage{booktabs}
\usepackage{longtable}
\usepackage{array}
\usepackage{multirow}
\usepackage{wrapfig}
\usepackage{float}
\usepackage{colortbl}
\usepackage{pdflscape}
\usepackage{tabu}
\usepackage{threeparttable}
\usepackage{threeparttablex}
\usepackage[normalem]{ulem}
\usepackage{makecell}
\usepackage{xcolor}
\usepackage{bookmark}
\IfFileExists{xurl.sty}{\usepackage{xurl}}{} % add URL line breaks if available
\urlstyle{same}
\hypersetup{
  hidelinks,
  pdfcreator={LaTeX via pandoc}}

\author{}
\date{\vspace{-2.5em}}

\begin{document}

\thispagestyle{firstpage}
\UNsection{SnazzieR Example Tables}{1.5}

\subsection{Imports}\label{imports}

\begin{Shaded}
\begin{Highlighting}[]
\FunctionTok{library}\NormalTok{(ggplot2)}
\FunctionTok{library}\NormalTok{(dplyr)}
\FunctionTok{library}\NormalTok{(gridExtra)}
\FunctionTok{library}\NormalTok{(grid)}
\FunctionTok{library}\NormalTok{(snazzieR)}
\FunctionTok{library}\NormalTok{(kableExtra)}
\end{Highlighting}
\end{Shaded}

\subsection{Iris Data Analysis}\label{iris-data-analysis}

\textbf{Source:} Fisher, R. (1936). Iris {[}Dataset{]}. UCI Machine
Learning Repository. \url{https://doi.org/10.24432/C56C76}.

\begin{Shaded}
\begin{Highlighting}[]
\FunctionTok{data}\NormalTok{(iris)}
\NormalTok{x.iris }\OtherTok{\textless{}{-}}\NormalTok{ iris[, }\FunctionTok{c}\NormalTok{(}\StringTok{"Sepal.Length"}\NormalTok{, }\StringTok{"Sepal.Width"}\NormalTok{)]}
\NormalTok{y.iris }\OtherTok{\textless{}{-}}\NormalTok{ iris[, }\FunctionTok{c}\NormalTok{(}\StringTok{"Petal.Length"}\NormalTok{, }\StringTok{"Petal.Width"}\NormalTok{)]}
\NormalTok{x.mat.iris }\OtherTok{\textless{}{-}} \FunctionTok{as.matrix}\NormalTok{(x.iris)}
\NormalTok{y.mat.iris }\OtherTok{\textless{}{-}} \FunctionTok{as.matrix}\NormalTok{(y.iris)}
\end{Highlighting}
\end{Shaded}

\begin{table}[H]
\centering
\caption{\label{tab:unnamed-chunk-3}Iris Dataset (First 6 Observations): Predictors and Responses}
\centering
\fontsize{12}{14}\selectfont
\begin{tabular}[t]{cccc}
\toprule
Sepal Length & Sepal Width & Petal Length & Petal Width\\
\midrule
\cellcolor{gray!10}{5.1} & \cellcolor{gray!10}{3.5} & \cellcolor{gray!10}{1.4} & \cellcolor{gray!10}{0.2}\\
4.9 & 3 & 1.4 & 0.2\\
\cellcolor{gray!10}{4.7} & \cellcolor{gray!10}{3.2} & \cellcolor{gray!10}{1.3} & \cellcolor{gray!10}{0.2}\\
4.6 & 3.1 & 1.5 & 0.2\\
\cellcolor{gray!10}{5} & \cellcolor{gray!10}{3.6} & \cellcolor{gray!10}{1.4} & \cellcolor{gray!10}{0.2}\\
5.4 & 3.9 & 1.7 & 0.4\\
\cellcolor{gray!10}{\vdots} & \cellcolor{gray!10}{\vdots} & \cellcolor{gray!10}{\vdots} & \cellcolor{gray!10}{\vdots}\\
\bottomrule
\end{tabular}
\end{table}

\subsubsection{Linear Regression
Analysis}\label{linear-regression-analysis}

\begin{Shaded}
\begin{Highlighting}[]
\CommentTok{\# Fit linear regression model}
\NormalTok{iris.lm }\OtherTok{\textless{}{-}} \FunctionTok{lm}\NormalTok{(Petal.Length }\SpecialCharTok{\textasciitilde{}}\NormalTok{ Sepal.Length }\SpecialCharTok{+}\NormalTok{ Sepal.Width, }\AttributeTok{data =}\NormalTok{ iris)}
\CommentTok{\# Model summary table}
\NormalTok{snazzieR}\SpecialCharTok{::}\FunctionTok{model.summary.table}\NormalTok{(iris.lm, }\AttributeTok{caption =} \StringTok{"Linear Regression Model Summary"}\NormalTok{)}
\end{Highlighting}
\end{Shaded}

\begin{table}[H]
\centering
\caption{\label{tab:unnamed-chunk-4}\textbf{ Linear Regression Model Summary } \newline \textbf{ Petal.Length = -2.525 + 1.776*Sepal.Length + -1.339*Sepal.Width }}
\centering
\begin{threeparttable}
\resizebox{\ifdim\width>\linewidth\linewidth\else\width\fi}{!}{
\begin{tabular}[t]{|>{}l|l|l|l|l||>{}l|>{}r|}
\hline
\textbf{Term} & \textbf{Estimate} & \textbf{Std.Error} & \textbf{P.Value} & \textbf{Signif.} & \textbf{Statistic} & \textbf{Value}\\
\hline
\textbf{(Intercept)} & -2.52476 & 0.56344 & 1e-05 & :3 & \textbf{MSE} & 0.40958\\
\hline
\textbf{Sepal.Length} & 1.77559 & 0.06441 & 0 & :3 & \textbf{MSE adj.} & 0.41794\\
\hline
\textbf{Sepal.Width} & -1.33862 & 0.12236 & 0 & :3 & \textbf{df} & 147.00000\\
\hline
\textbf{} &  &  &  &  & \textbf{R-squared} & 0.86769\\
\hline
\textbf{} &  &  &  &  & \textbf{R-squared adj.} & 0.86589\\
\hline
\end{tabular}}
\begin{tablenotes}[para]
\item 
\item 
\item 
\item significance codes -  :3 -  >0.001 
\item :) - >0.01
\item :/ - >0.05 
\item 
\end{tablenotes}
\end{threeparttable}
\end{table}

\subsubsection{Model Equation}\label{model-equation}

\begin{Shaded}
\begin{Highlighting}[]
\NormalTok{snazzieR}\SpecialCharTok{::}\FunctionTok{model.equation}\NormalTok{(iris.lm)}
\end{Highlighting}
\end{Shaded}

\[\text{Petal.Length} = -2.525 + 1.776 (\text{Sepal.Length}) - 1.339 (\text{Sepal.Width})\]

\subsubsection{ANOVA Analysis}\label{anova-analysis}

\begin{Shaded}
\begin{Highlighting}[]
\NormalTok{snazzieR}\SpecialCharTok{::}\FunctionTok{ANOVA.summary.table}\NormalTok{(iris.lm, }\AttributeTok{caption =} \StringTok{"ANOVA Results"}\NormalTok{)}
\end{Highlighting}
\end{Shaded}

\begin{table}[H]
\centering
\caption{\label{tab:unnamed-chunk-6}\textbf{ANOVA Results}}
\centering
\begin{threeparttable}
\begin{tabular}[t]{|>{}l|r|r|r|l|l|>{}l|}
\hline
\textbf{Term} & \textbf{Df} & \textbf{Sum.Sq} & \textbf{Mean.Sq} & \textbf{F.Value} & \textbf{P.Value} & \textbf{Signif.}\\
\hline
\textbf{Sepal.Length} & 1 & 352.86624 & 352.86624 & 844.30476 & 0 & :3\\
\hline
\textbf{Sepal.Width} & 1 & 50.02241 & 50.02241 & 119.68886 & 0 & :3\\
\hline
\textbf{Residuals} & 147 & 61.43675 & 0.41794 &  &  & :3\\
\hline
\end{tabular}
\begin{tablenotes}[para]
\item 
\item 
\item 
\item significance codes -  :3 -  >0.001 
\item :) - >0.01
\item :/ - >0.05 
\item 
\end{tablenotes}
\end{threeparttable}
\end{table}

\subsubsection{Eigenvalue Analysis}\label{eigenvalue-analysis}

\begin{Shaded}
\begin{Highlighting}[]
\CommentTok{\# Prepare iris data and standardize}
\NormalTok{iris.data }\OtherTok{\textless{}{-}}\NormalTok{ iris[, }\DecValTok{1}\SpecialCharTok{:}\DecValTok{4}\NormalTok{]}
\NormalTok{scaled.data }\OtherTok{\textless{}{-}} \FunctionTok{as.data.frame}\NormalTok{(}
  \FunctionTok{lapply}\NormalTok{(iris.data, }\ControlFlowTok{function}\NormalTok{(x) \{}
\NormalTok{    (x }\SpecialCharTok{{-}} \FunctionTok{mean}\NormalTok{(x)) }\SpecialCharTok{/} \FunctionTok{sd}\NormalTok{(x)}
\NormalTok{  \})}
\NormalTok{)}
\NormalTok{correlation.matrix }\OtherTok{\textless{}{-}} \FunctionTok{cor}\NormalTok{(scaled.data)}
\CommentTok{\# Eigenvalue analysis}
\NormalTok{snazzieR}\SpecialCharTok{::}\FunctionTok{eigen.summary}\NormalTok{(correlation.matrix)}
\end{Highlighting}
\end{Shaded}

\begin{table}[!h]
\centering
\caption{\label{tab:unnamed-chunk-7}Eigenvectors of Covariance Matrix}
\centering
\begin{threeparttable}
\begin{tabular}[t]{cccc}

\multicolumn{1}{c}{ } \\
$\lambda_1 = 2.9185$ & $\lambda_2 = 0.914$ & $\lambda_3 = 0.1468$ & $\lambda_4 = 0.0207$\\
\arrayrulecolor{white}
\midrule
$\begin{bmatrix}0.52107\\-0.26935\\0.58041\\0.56486\end{bmatrix}$ & $\begin{bmatrix}0.37742\\0.9233\\0.02449\\0.06694\end{bmatrix}$ & $\begin{bmatrix}0.71957\\-0.24438\\-0.14213\\-0.63427\end{bmatrix}$ & $\begin{bmatrix}-0.26129\\0.12351\\0.80145\\-0.5236\end{bmatrix}$\\
\bottomrule
\end{tabular}
\begin{tablenotes}[para]
\item Total Variance = 4
\end{tablenotes}
\end{threeparttable}
\end{table}

\subsubsection{PLS Regression (NIPALS)}\label{pls-regression-nipals}

\begin{Shaded}
\begin{Highlighting}[]
\NormalTok{NIPALS.pls.iris }\OtherTok{\textless{}{-}}\NormalTok{ snazzieR}\SpecialCharTok{::}\FunctionTok{pls.regression}\NormalTok{(x.mat.iris, y.mat.iris, }\AttributeTok{n.components =} \DecValTok{3}\NormalTok{, }\AttributeTok{calc.method =} \StringTok{"NIPALS"}\NormalTok{)}
\NormalTok{snazzieR}\SpecialCharTok{::}\FunctionTok{pls.summary}\NormalTok{(NIPALS.pls.iris, }\AttributeTok{include.scores =} \ConstantTok{FALSE}\NormalTok{)}
\end{Highlighting}
\end{Shaded}

\begin{minipage}[t]{0.48\linewidth}
\begin{table}[H]
\centering
\caption{\label{tab:unnamed-chunk-8}X Weights (W)}
\centering
\begin{tabular}[t]{|>{}c>{}c|}
\toprule
Comp 1 & Comp 2\\
\midrule
-0.9046320 & 0.4261936\\
0.4261936 & 0.9046320\\
\bottomrule
\end{tabular}
\end{table}
\end{minipage}\begin{minipage}[t]{0.48\linewidth}
\begin{table}[H]
\centering
\caption{\label{tab:unnamed-chunk-8}Y Weights (C)}
\centering
\begin{tabular}[t]{|>{}c>{}c|}
\toprule
Comp 1 & Comp 2\\
\midrule
-0.7350032 & 0.5400163\\
-0.6780636 & 0.8416546\\
\bottomrule
\end{tabular}
\end{table}
\end{minipage}

\begin{minipage}[t]{0.48\linewidth}
\begin{table}[H]
\centering
\caption{\label{tab:unnamed-chunk-8}X Loadings (P)}
\centering
\begin{tabular}[t]{|>{}c>{}c|}
\toprule
Comp 1 & Comp 2\\
\midrule
-11.159218 & 4.946903\\
6.224581 & 10.500219\\
\bottomrule
\end{tabular}
\end{table}
\end{minipage}\begin{minipage}[t]{0.48\linewidth}
\begin{table}[H]
\centering
\caption{\label{tab:unnamed-chunk-8}Y Loadings (Q)}
\centering
\begin{tabular}[t]{|>{}c>{}c|}
\toprule
Comp 1 & Comp 2\\
\midrule
-0.7350032 & 0.5400163\\
-0.6780636 & 0.8416546\\
\bottomrule
\end{tabular}
\end{table}
\end{minipage}

\begin{table}[H]
\centering
\caption{\label{tab:unnamed-chunk-8}Regression Scalars (b)}
\centering
\begin{tabular}[t]{|>{}c>{}c|}
\toprule
Component & Estimate\\
\midrule
1 & 15.444539\\
2 & 1.203207\\
\bottomrule
\end{tabular}
\end{table}

\begin{table}[H]
\centering
\caption{\label{tab:unnamed-chunk-8}Regression Coefficients (Original Scale)}
\centering
\begin{tabular}[t]{|>{}lc>{}c|}
\toprule
  & Petal.Length & Petal.Width\\
\midrule
Sepal.Length & 1.775592 & 0.7232920\\
Sepal.Width & -1.338623 & -0.4787213\\
\bottomrule
\end{tabular}
\end{table}

\begin{table}[H]
\centering
\caption{\label{tab:unnamed-chunk-8}Variance Explained by Components (X)}
\centering
\resizebox{\ifdim\width>\linewidth\linewidth\else\width\fi}{!}{
\begin{tabular}[t]{|>{\centering\arraybackslash}p{3cm}c>{}c|}
\toprule
Latent Vector & Explained Variance & Cumulative\\
\midrule
1 & 54.7898\% & 54.7898\%\\
2 & 45.2102\% & 100.0000\%\\
\bottomrule
\end{tabular}}
\end{table}

\begin{table}[H]
\centering
\caption{\label{tab:unnamed-chunk-8}Variance Explained by Components (Y)}
\centering
\resizebox{\ifdim\width>\linewidth\linewidth\else\width\fi}{!}{
\begin{tabular}[t]{|>{\centering\arraybackslash}p{3cm}c>{}c|}
\toprule
Latent Vector & Explained Variance & Cumulative\\
\midrule
1 & 80.0449\% & 80.0449\%\\
2 & 0.4858\% & 80.5307\%\\
\bottomrule
\end{tabular}}
\end{table}

\end{document}
